\documentclass{tycn_art}
\usepackage{ty_maths}
\usepackage{listings}
\lstset{language=[LaTeX]TeX}
\lstset{numbers=left,numberstyle=\tiny}
\lstset{basicstyle=\ttfamily\small,keywordstyle=\bfseries}
\lstset{frame=single,backgroundcolor=\color{AliceBlue}}
\lstset{showstringspaces=false,escapechar=§,columns=fixed}

\fancyhf{}
\fancyhead[L]{\LaTeX 超快入门教程}
\fancyhead[R]{\itshape\leftmark}
\fancyfoot[C]{\thepage}
\fancypagestyle{plain}{
\fancyhf{}
\fancyhead[L]{\LaTeX 超快入门教程}
\fancyhead[R]{\itshape\leftmark}
\fancyfoot[C]{\thepage}}

\palatino
\pdftac{LaTeX超快入门教程}{李天意}{RoyalBlue}
\title{\LaTeX 超快入门教程}
\author{李天意}

\begin{document}
\maketitle

\section{它到底是啥}
\LaTeX 本质上来说是一种语言,但与其它编程语言不同,它写的源代码将被编译成一篇文章或一本书,而不是一个程序。可能大家更习惯用类似Word的软件进行文档的编写,那么它们之间的区别是什么?为什么会产生 \LaTeX ?

事实上,先产生的是 \TeX 语言。其在当时之所以诞生,是因为当时的发明者大牛 \href{http://en.wikipedia.org/wiki/Donald\_Knuth}{Donald Knuth} 在交给出版社让其编辑出版自己写的一本书的时候,发现文章样式尤其是数学公式的排版尤其糟糕,于是就萌发了自己设计一个排版语言软件的想法,于是就产生了 \TeX 语言。\TeX 语言比较初级,作为非信息的学生可能下面这个比喻不好,就好像汇编语言(接近机器语言)那样;于是之后就产生了 \LaTeX ,相当于一个再封包,增加许多比较高级的命令,相对汇编 \LaTeX 就类似于 C 语言。

正如刚才所说,\TeX 或 \LaTeX 语言的诞生(下面都直接说 \LaTeX,意指所有从 \TeX 中衍生出来的科技文档编写语言)主要是为了排版带有数学公式的文档。你可能会说 Word 也可以打数学啊,我用 Mathtype 打公式我也挺好么?的确,直接 Word(比如2007版的公式编辑功能)或配合其他插件(如 Mathtype)也\emph{可以}打公式,但是从美观这个指标来看,远远没有 \LaTeX 制作的数学公式在字体运用、字符各个指标的位置及各个数学量之间的间隙控制得好,这一点可以自己Google搜索``Word, LaTeX''看看,尽管我承认现在 Word 就\emph{公式而言}美观程度是在逐渐随着版本的提高而改善。

我刚才说到就\emph{公式而言},是因为 \LaTeX 设计的理念是,将排版和内容基本上分开管理:排版,指的是正文字体是啥啦、多大啦、标题是什么字体啦、每个章节是什么格式啦、章节编号如何编号啦、段与段之间间隔多少啦、公式与普通文字之间的间隔如何控制啦\ldots\ldots 而且,就我而言60\%的 \LaTeX 及已加载宏包(后面提到)的默认设置就已经很好了,也就是说排版并不需要占你写这篇文章或这本书很大的部分,你只要专心于内容就好了,这就是 \LaTeX 与 Word 最本质的区别。

在讲 Word 之前,我首先要承认 Word 的确具有将排版与内容想分开管理的模式,但其管理得并不严密,因为 Word 制作文档的理念是所见即所得,也就是说作者仍然可以或需要修改或设计排版。撇开``排版与内容分开管理''不说,\LaTeX 能最发挥其特色的地方还是体现在科技文档的系统排版上。科技文档,当然需要一个相对固定的格式(尤其是章节层次),通常定下来之后就不会再发生什么很大的变化。再加上科技文档中的数学公式比较多,\LaTeX 可以相对较好地控制页面的留白处,做到数学与文字的\emph{和谐}(这个词不用取引申含义)。

举一个 \LaTeX 文档的大致框架的例子,就这篇文章好了:
\begin{lstlisting}
\documentclass{tycn_art}
\usepackage{tymaths}
\usepackage{listings}
\palatino
\pdftac{LaTeX§超快入门教程§}{§李天意§}{Blue}
\title{\LaTeX §超快入门教程§}
\author{§李天意§}

\begin{document}
\maketitle
\section{§它到底是啥§}

....

\end{document}
\end{lstlisting}

第1行的 \verb|\documentclass{tycn_art}| 为定义文档的类型,即到底是一篇小文章还是一篇报告或一本书。类型 \verb|tycn_art| 是我自己设置的专门为了写中文文章的类型,添加了许多针对中文的修改,一劳永逸,之后只要加载这个类型就可以了。

第2行的 \verb|\usepackage{tymaths}| 为加载我自己设置的数学宏包。即加载了一些写数学用到的宏包和定义一些可能会用到的命令,下面会提到。第3行的 \verb|listings|是用来显示各种编程源代码的宏包,你看到的上面那个灰框框就是它做出来的。由于这个宏包不是经常用,所以我没有放在自己设置文档类型中(当然更没有放在我自设的数学宏包中,因为这个和数学没啥关系),所以我单独加载。

第4行的 \verb|\palatino| 是我自己在文档类型宏包中定义的命令,意思是使用 Palatino 字体,一个我很喜欢的字体,本文的英文和数学字体就是它,当然各种源代码采用等款字体\texttt{Consolas}.

第5行的 \verb|\pdftac| 也是我自己定义的一个命令,意思是设置 PDF 文档的属性,在这里设置了标题 T、作者 A和超链接的颜色 C。第6行和第7行设置本文章的标题和作者信息。第1到第7行称作 \LaTeX 文章的导言区,因为下面就开始正文了,由第9行的 \verb|\begin{document}| 作为标志。

开始正文后,第10行的 \verb|\maketitle| 进行制作``封面'',即打印标题、作者和时间信息。随后的 \verb|\section| 命令就开始新建一个章节,开始正文。再之后如果要再开一个章节,就再 \verb|\section| 即可。文档的最后用 \verb|\end{document}| 标志。

如果上手 \LaTeX 貌似是一个初心者要琢磨很长时间的问题,我记得当时在白天中也摸索也很久。说白天是因为网上资料真得很多,但问题是近来 \LaTeX 发展很快,一些陈旧的方法已经不再实用,这样新老文章在一起也不知道到底应该选择什么\ldots 其实有点过了,因为其实大致 \LaTeX 的框架已经固定下来,变的,只是一些宏包和选项。下面开始细讲。

\section{名词解释}
前面已经提到很多 \LaTeX 的专有名词,下面一一来解释。
\begin{description}
\item[文档类型] 指的是你目前写的文档到底是一个什么类型(废话),\LaTeX 的标准类有:文章类 \texttt{article},报告类 \texttt{report} 和书籍类 \texttt{book},一般来说够用了,不过像我写动机信还会用到 \texttt{lettre} 类等等。在 \LaTeX 源代码中用 \verb|\documentclass{class}| 来指明所用文档类。你既可以用已经设置好的标准类,也可以用其他宏包提供的类型(如 \texttt{CTeX} 宏包提供的专门写中文的文章类 \texttt{ctexart} 等),当然也可以自己设置一个文档类。
\item[宏包] 宏包就类似于一个包括了很多设置和新命令的大礼包,当你需要什么的时候就去加载一个合适的宏包。比如如果会用到一些数学符号比如要输入矩阵,那么通常就加载 \texttt{amsmath} 宏包,它提供了 \verb|bmatrix| 环境就用来输入矩阵。我自己也做了几个宏包,其实就是把自己平时常常会用到的命令放到一个文件(宏包)之中,以后需要用到只要加载这个宏包即可。宏包的选择是入门 \LaTeX 的一个难点\ldots 因为不知道到底什么宏包好,什么不好。调用宏包使用 \verb|\usepackage| 命令。
\item[环境] 在正文中可能会使用到某些环境,比如你想将一段文章居中排,那么可以用到 \texttt{center} 环境
\begin{lstlisting}
\begin{center}
§你想要居中的东西§
\end{center}
\end{lstlisting}
来实现。环境就是一个 \texttt{begin} 一个 \texttt{end},放在里面的东西就处于这个环境之下,具有特殊的排版效果。事实上写这篇文章的时候,上面你看到的这个放源代码的灰底框架就是使用了 \texttt{listings} 宏包提供的 \verb|lstlisting| 环境,在这个环境中的东西自动会编程源代码的样子。
\item[命令] \ldots 所谓命令其实就是 \LaTeX 的语法,\LaTeX 本身以及你加载的文档类、宏包都会告诉你它都定义了哪些命令供你使用。\LaTeX 的命令都由反斜杠 \verb|\| 开头,且全英文不允许数字。比如我要输出 LaTeX 的标志,我就输入 \verb|\LaTeX|,编译后就成了 \LaTeX。你也可以自己定义一个命令,因为可能这个命令会经常使用。
\item[导言区] 所有在 \verb|\begin{document}| 之前的命令都处于导言区,在导言区中我们定义文档类型,加载需要用到的宏包和设置一些属性、定义自己的命令。
\end{description}

\section{下载LaTeX处理程序}
\LaTeX 只是一种编程语言,所以我们需要编辑器和编译器,好在现在 \LaTeX 提供了很多发行版,意思就是将编译器、很多实用宏包、编辑器和其他辅助软件等捆绑在一起发行。

Windows 用户推荐大家下载 \href{http://miktex.org/}{MiKTeX},其宏包管理系统``飞行下载''很实用,即当你编译的时候发现你需要调用的那个宏包还没有下载(还没有在你电脑上),它会自动帮你去下载。它默认使用 Texworks 作为编辑器,个人认为是一个小巧但实用的编译器,且默认编码为 UTF-8,适合打中英法文。

Mac 用户推荐大家下载 \href{http://www.tug.org/mactex/2011/}{MacTeX},事实上它是基于 TexLive 发行版的,但再加上了几个适合苹果的辅助软件。编辑器推荐大家用 TeXShop,很漂亮而且也很实用,事实上 Texworks 就是根据它向 Windows 迁移的。

\section{处理中文}
\TeX  发明得很早,在那时没有怎么考虑怎么整合亚洲文字\ldots 好在现在已经有一套完整且较完美的解决方案了。

处理中文的第一步是选择编译语言。大家可能会觉得奇怪,说了那么多 \LaTeX 难道编译语言不是TA么?事实上是又不是。因为如今基于 \LaTeX 的新语言很多,它们往往更实用,于是直接采用 \LaTeX 进行编译的情况基本上没有。在这里我推荐大家使用 \texttt{XeLaTeX} 进行编译(可以在编辑器中选择),这个语言有 \LaTeX 提供的新命令(相比 \TeX),而且能够很好地支持多语言,比如一会中文一会法语,其默认编码也为 UTF-8。基于 \texttt{XeLaTeX} 也诞生了很多宏包,比如下面会提到的处理中文的 \texttt{CTeX} 宏包。

在 \texttt{XeLaTeX} 编译语言下,\texttt{CTeX} 宏包将配合专门用来设置中文的 \texttt{xeCJK} 宏包一起来调整中文文章的排版效果。如果你要写一篇有中文的文档,那么只需要加载 \texttt{CTeX} 宏包提供的 \texttt{ctexart}, \texttt{ctexrep} 和 \texttt{ctexbook} 文档类即可,一般它会自动调整字体并成功显示中文。不过有时候,比如在 Mac 下,可能需要自己设置,那么其实也很简单,只需要在导言区加上比如
\begin{lstlisting}
\setCJKmainfont[BoldFont=SimHei,ItalicFont=KaiTi]{SimSun}
\setCJKmonofont{FangSong}
\end{lstlisting}
之类的命令即可,意思就是将亚洲语言的字体默认设置为宋体,其中粗体采用黑体、斜体采用楷体、等宽字体(源代码字体)采用仿宋。当然前提是你电脑上要有这些字体,Windows 用户可能没问题,Mac 的话可以拷\ldots\ldots

\section{数学}
正如前面所言,\LaTeX 一个很大的优势就是数学公式。我们可以使用行内公式$f:\Icc{0,1}\to\fd{C}$,也可以使用独立公式
\[
J_f(x)\eqdef\rp{\md f_x}=\begin{bmatrix}
\difp{f_1}{x_1}(x) & \difp{f_1}{x_2}(x) & \cdots & \difp{f_1}{x_n}(x) \\
\vdots & \vdots & & \vdots \\
\difp{f_m}{x_1}(x) & \difp{f_m}{x_2}(x) & \cdots & \difp{f_m}{x_n}(x)
\end{bmatrix}
\]
个人觉得比 Word 啥的好看多了,不是么\ldots\ldots

行内公式用 \verb|文字$公式$文字| 来实现,即用美元符号包住公式\footnote{原因是发明者认为数学很贵\ldots\ldots}。独立公式根据情况的不同可以采用不同的环境。上面的用
\begin{lstlisting}
\[
J_f(x)\eqdef\rp{\md f_x}=\begin{bmatrix}
\difp{f_1}{x_1}(x) & \difp{f_1}{x_2}(x) & \cdots & \difp{f_1}{x_n}(x) \\
\vdots & \vdots & \vdots & \vdots \\
\difp{f_m}{x_1}(x) & \difp{f_m}{x_2}(x) & \cdots & \difp{f_m}{x_n}(x)
\end{bmatrix}
\]
\end{lstlisting}
来实现,是不会显示公式编号的。如果用
\begin{lstlisting}
\begin{equation}
\sum_{n=-\infty}^{+\infty}\hat{f}(n/T)=T\sum_{n=-\infty}^{+\infty}f(nT)
\end{equation}
\end{lstlisting}
那么效果就是
\begin{equation}
\sum_{n=-\infty}^{+\infty}\hat{f}(n/T)=T\sum_{n=-\infty}^{+\infty}f(nT)
\end{equation}

当然,还有很多情况,比如多行公式。我这里就不细讲了,大家可以参考 mathmode 或 The LaTeX Mathematics Companion 这两本书。

\section{写在后面}
这篇文章实在是太简略了,在下载、使用 \LaTeX 的过程中会遇到无数个问题,但本文可能一个都无法回答。不是因为我懒(可能也是),而是因为问题太多、太杂,而解决方案层出不穷。不管如何,给出学习 \LaTeX 旅程中下面的路。

多逛逛 \href{http://bbs.ctex.org/}{CTeX 论坛},看看里面的置顶贴,如果有疑问也可以在里面发帖。如果我不小心看到你发的一个傻问题我会好心地回答的。 里面也分享了很多教程,不管是一般的还是针对某个特定主题的(比如数学)。其中 lshort 这本书在我一开始学的时候帮了不少忙,可以看看这本书。

希望大家学会用 Google,上面提到的所有书都可以搜索到。如果有什么私人问题,如果相信我的回复速度的话,那么就给我发邮件,最好不要什么太傻的问题\ldots\ldots
\begin{center}
\texttt{tianyikillua(at)gmail.com}
\end{center}
\end{document}